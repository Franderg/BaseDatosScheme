\documentclass[12pt,a4paper]{report}
\usepackage[utf8]{inputenc}
\makeatletter

%Frander Granados Vega
%201117284
%Ingenier\'i­a en Computadores

\pdfinfo{%
  /Title    (Documentaci\'{o}n)
  /Author   (Frander Granados Vega)
}


\begin{document}
\begin{center}
{\LARGE Tecnol\'{o}gico de Costa Rica \\[2 cm]}

{\Large Lenguajes, compiladores e intérpretes \\[3 cm]}

{\Large Ingenier\'{i}a en Computadores \\[3 cm]}

{\Large Primer Proyecto programado \\[3 cm]}

{\large Estudiantes: \\[0.3 cm] }
{\Large Frander Granados 201117284 \\}
{\Large Maikol Barrantes 201230364\\[3 cm]}

{\Large I Semestre, 2015 \\[2 cm] }
\end{center}


\newpage

\begin{flushleft}

1. Manual de usuario: cómo ejecutar el programa, incluir requerimientos de
Hardware y Software

\end{flushleft}

\begin{flushleft}

2. Descripción de las estructuras de datos desarrolladas.

\end{flushleft}

\begin{flushleft}
3. Descripción de las principales funciones desarrollados.\\\

Se define un shell llamado consola, (define (consola), esta función tiene un read-line que se guarda en una variable de entrada, dicha variable es analizada
para verificar si es alguno de los comandos definidos.\\
Tiene una condición de parada cuando se escriba "exit".\\
En caso de ingresar un comando valido el programa procede a realizar la operación, verificando primero que la entrada sea string.\\
Este shell le pasa lo captado a una funcion verifica que como dice el nombre nos verifica el comando usado y procede a realizarlo.\\
Para saber que comando se usa "(let ((comando (car(string-split ls))))" con esto se verifica la variable comando que al usar el string-split nos hace la linea
en palabras separadas ejemplo: "hola esto es una prueba"  al aplicarle el string-split $\rightarrow$ "hola" "esto" "es" "una" "prueba" , ya con eso le hacemos car a 
la variable y nos dará el primer elemento, que para el caso de nuestro programa debería ser el comando a usar.\\\

Comando ct:\\
Para este comando se tiene el primer elemento después del comando ct debe ser el nombre de la tabla, para nuestro caso va a ser el nombre del archivo.\\
Se tiene la función y se verifica si va a contener algo o no, en caso de que no vaya a contener nada se crea el archivo vacío, si se van a agregar columnas
se tomas los datos y se pasan a una función auxiliar recursiva, la función escribe en el archivo los datos separados por dos puntos ":", que va a ser nuestro
separador para facilitar las cosas.\\\

Comando isn:\\
Para este comando se debe hacer una validación especial, esto porque el comando puede venir de dos maneras.\\
En la primera se indica la tabla en la que se va a insertar información, y se incluyen en orden datos para todas y cada una de las columnas de esa tabla, como por ejemplo:
"ins estud 2012001 julio 5554444 ", este caso no tiene nada especial.\\
En el segundo formato se indica por medio de una lista el nombre de las columnas que corresponden a los datos incluidos. No es necesario incluir datos para todas las
columnas, pero sí debe haber un dato para la llave como por ejemplo: "ins estud (nombre carnet) maria 2010002 "\\
Para implementar este segundo, se crea una verificación guardando en una variable el tercer elemento (let ((verif(car(cdr(cdr( string-split entrada )))))), el primero corresponde al comando, el segundo a la llave y el tercero puede ser una columna o en este segundo caso es una lista y la sintaxis del comando indica que tiene paréntesis. Tomando esa palabra con el uso de string-ref podemos obtener el elementos que queramos de un string, esta función pasa a char la palabra y nos devuelve el elemento que le indiquemos, ejemplo: la palabra "heisenberg" le aplicamos el string-ref y se obtiene $\#$h $\#$e $\#$i $\#$s $\#$e $\#$n $\#$b $\#$e $\#$r $\#$g




\end{flushleft}

\begin{flushleft}

4. Problemas conocidos: En está sección se detalla cualquier problema que no se ha podido solucionar en el trabajo.

\end{flushleft}

\begin{flushleft}
5. Actividades realizadas por estudiante: Este es un resumen de las bitácoras
de cada estudiante ( estilo timesheet) en términos del tiempo invertido para
una actividad específica que impactó directamente el desarrollo del trabajo,
de manera breve (no más de una línea) se describe lo que se realizó, la
cantidad de horas invertidas y la fecha en la que se realizó. Se deben sumar
las horas invertidas por cada estudiante, sean conscientes a la hora de
realizar esto el profesor determinará si los reportes están acordes al
producto entregado.

\end{flushleft}

\begin{flushleft}

6. Corridas de ejemplo: Mostrar los resultados obtenidos al correr la secuencia
de comandos que se muestra en la página anterior. En caso de
implementación parcial de la tarea mostrar la ejecución de aquellas partes
del programa que funcionan.

\end{flushleft}

\begin{flushleft}

7. Problemas encontrados: descripción detallada, intentos de solución sin éxito, solución encontradas con su descripción detallada, recomendaciones, conclusiones y bibliografía consultada para este problema específico.

Un problema encontrado fue a la hora de escribir en archivos, el problema es que con la función creada por alguna razón al pasarle un newline 
en lugar de hacer el "$\backslash$n" escribia el proceso newline, ejemplo "\(<\)$\#$procedure:newline\(>\)"\\
Para poder hacer el "enter" al final de cada linea se crea otra función para hacerlo.\\
 


\end{flushleft}

\begin{flushleft}

8. Conclusiones y Recomendaciones del proyecto.

\end{flushleft}

\begin{flushleft}

9. Bibliografía consultada en todo el proyecto




Documentación Racket

Benson, B. (s. f.). 4.3 Strings Recuperado de http://docs.racket-lang.org/reference/strings.html

Benson, B. (s. f.). 4.5 Characters Recuperado de http://docs.racket-lang.org/reference/characters.html$\#$\%$28def.$\_$\%28$\%28quote.$\_$~23$\sim$25kernel$\%$29.$\_$char
$\sim3$f$\%$29$\%$29

Getting a line of user input in Scheme? - Stack Overflow (2010, 04 de Julio). Recuperado de https://stackoverflow.com/questions/3173327/getting-a-line-of-user-input-in-scheme


demas (2011, 07 de Octubre). Scheme - String split function - Stack Overflow Recuperado el 30 de Mayo del 2015, de https://stackoverflow.com/questions/7691769/string-split-function


artofproblemsolving (s. f.). LaTeX:Symbols Recuperado de http://www.artofproblemsolving.com/wiki/index.php/LaTeX%3ASymbols








\end{flushleft}


\end{document}