\documentclass[12pt,a4paper]{report}
\usepackage[utf8]{inputenc}
\makeatletter

%Frander Granados Vega
%201117284
%Ingenier\'i­a en Computadores

\pdfinfo{%
  /Title    (Documentaci\'{o}n)
  /Author   (Frander Granados Vega)
}


\begin{document}
\begin{center}
{\LARGE Tecnol\'{o}gico de Costa Rica \\[2 cm]}

{\Large Lenguajes, compiladores e intérpretes \\[3 cm]}

{\Large Ingenier\'{i}a en Computadores \\[3 cm]}

{\Large Primer Proyecto programado \\[3 cm]}

{\large Estudiantes: \\[0.3 cm] }
{\Large Frander Granados 201117284 \\}
{\Large Maikol Barrantes 201230364\\[3 cm]}

{\Large I Semestre, 2015 \\[2 cm] }
\end{center}


\newpage

\begin{flushleft}

1. Manual de usuario: cómo ejecutar el programa, incluir requerimientos de
Hardware y Software

\end{flushleft}

\begin{flushleft}

2. Descripción de las estructuras de datos desarrolladas.

\end{flushleft}

\begin{flushleft}
3. Descripción de las principales funciones desarrollados.\\\

Se define un shell llamado consola, (define (consola), esta función tiene un read-line que se guarda en una variable de entrada, dicha variable es analizada
para verificar si es alguno de los comandos definidos.\\
Tiene una condición de parada cuando se escriba "exit".\\
En caso de ingresar un comando valido el programa procede a realizar la operación, verificando primero que la entrada sea string.\\
Para saber que comando se usa "(let ((comando (car(string-split ls))))" con esto se verifica la variable comando que al usar el string-split nos hace la linea
en palabras separadas ejemplo: "hola esto es una prueba"  al aplicarle el string-split $->$ "hola" "esto" "es" "una" "prueba" , ya con eso le hacemos car a 
la variable y nos dará el primer elemento, que para el caso de nuestro programa debería ser el comando a usar.\\\

Comando ct:\\
Para este comando se tiene el primer elemento después del comando ct debe ser el nombre de la tabla, para nuestro caso va a ser el nombre del archivo.\\
Se tiene la función y se verifica si va a contener algo o no, en caso de que no vaya a contener nada se crea el archivo vacío, si se van a agregar columnas
se tomas los datos y se pasan a una función auxiliar recursiva, la función escribe en el archivo los datos separados por dos puntos ":", que va a ser nuestro
separador para facilitar las cosas.\\\

Comando isn:


\end{flushleft}

\begin{flushleft}

4. Problemas conocidos: En está sección se detalla cualquier problema que no se ha podido solucionar en el trabajo.

\end{flushleft}

\begin{flushleft}
5. Actividades realizadas por estudiante: Este es un resumen de las bitácoras
de cada estudiante ( estilo timesheet) en términos del tiempo invertido para
una actividad específica que impactó directamente el desarrollo del trabajo,
de manera breve (no más de una línea) se describe lo que se realizó, la
cantidad de horas invertidas y la fecha en la que se realizó. Se deben sumar
las horas invertidas por cada estudiante, sean conscientes a la hora de
realizar esto el profesor determinará si los reportes están acordes al
producto entregado.

\end{flushleft}

\begin{flushleft}

6. Corridas de ejemplo: Mostrar los resultados obtenidos al correr la secuencia
de comandos que se muestra en la página anterior. En caso de
implementación parcial de la tarea mostrar la ejecución de aquellas partes
del programa que funcionan.

\end{flushleft}

\begin{flushleft}

7. Problemas encontrados: descripción detallada, intentos de solución sin éxito,
solución encontradas con su descripción detallada, recomendaciones,
conclusiones y bibliografía consultada para este problema específico.

\end{flushleft}

\begin{flushleft}

8. Conclusiones y Recomendaciones del proyecto.

\end{flushleft}

\begin{flushleft}

9. Bibliografía consultada en todo el proyecto




Documentación Racket

Benson, B. (s. f.). 4.3 Strings Recuperado de http://docs.racket-lang.org/reference/strings.html

Benson, B. (s. f.). 4.5 Characters Recuperado de http://docs.racket-lang.org/reference/characters.html$\#$\%$28def.$\_$\%28$\%28quote.$\_$~23$\sim$25kernel$\%$29.$\_$char
$\sim3$f$\%$29$\%$29

Getting a line of user input in Scheme? - Stack Overflow (2010, 04 de Julio). Recuperado de https://stackoverflow.com/questions/3173327/getting-a-line-of-user-input-in-scheme


demas (2011, 07 de Octubre). Scheme - String split function - Stack Overflow Recuperado el 30 de Mayo del 2015, de https://stackoverflow.com/questions/7691769/string-split-function











\end{flushleft}


\end{document}